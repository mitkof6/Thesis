%%%%%%%%%%%%%%%%%%%%%%%%%%%%%%%%%%%%%%%%%
% Beamer Presentation
% LaTeX Template
% Version 1.0 (10/11/12)
%
% This template has been downloaded from:
% http://www.LaTeXTemplates.com
%
% License:
% CC BY-NC-SA 3.0 (http://creativecommons.org/licenses/by-nc-sa/3.0/)
%
%%%%%%%%%%%%%%%%%%%%%%%%%%%%%%%%%%%%%%%%%

%----------------------------------------------------------------------------------------
%	PACKAGES AND THEMES
%----------------------------------------------------------------------------------------

\documentclass[8pt,sans,mathserif,aspectratio=43]{beamer}%aspectratio=43

\usepackage{ucs}
\usepackage[T1]{fontenc}
\usepackage[utf8x]{inputenc}
\usepackage[english, greek]{babel}
%\usepackage{kerkis}

\mode<presentation> {

% The Beamer class comes with a number of default slide themes
% which change the colors and layouts of slides. Below this is a list
% of all the themes, uncomment each in turn to see what they look like.

%\usetheme{default}
%\usetheme{AnnArbor}
%\usetheme{Antibes}%good
%\usetheme{Bergen}
%\usetheme{Berkeley}
%\usetheme{Berlin}
%\usetheme{Boadilla}
%\usetheme{CambridgeUS}%good red
%\usetheme{Copenhagen}
\usetheme{Darmstadt}%good margin
%\usetheme{Dresden}
%\usetheme{Frankfurt}
%\usetheme{Goettingen}
%\usetheme{Hannover}
%\usetheme{Ilmenau}
%\usetheme{JuanLesPins}%good
%\usetheme{Luebeck}
%\usetheme{Madrid}
%\usetheme{Malmoe}
%\usetheme{Marburg}
%\usetheme{Montpellier}
%\usetheme{PaloAlto}
%\usetheme{Pittsburgh}
%\usetheme{Rochester}
%\usetheme{Singapore}
%\usetheme{Szeged}
%\usetheme{Warsaw}

% As well as themes, the Beamer class has a number of color themes
% for any slide theme. Uncomment each of these in turn to see how it
% changes the colors of your current slide theme.

%\usecolortheme{albatross}
%\usecolortheme{beaver}
%\usecolortheme{beetle}
%\usecolortheme{crane}
%\usecolortheme{dolphin}
%\usecolortheme{dove}
%\usecolortheme{fly}
%\usecolortheme{lily}
%\usecolortheme{orchid}
%\usecolortheme{rose}
%\usecolortheme{seagull}
%\usecolortheme{seahorse}
%\usecolortheme{whale}
%\usecolortheme{wolverine}

%\setbeamertemplate{footline} % To remove the footer line in all slides uncomment this line
%\setbeamertemplate{footline}[page number] % To replace the footer line in all slides with a simple slide count uncomment this line

%\setbeamertemplate{navigation symbols}{} % To remove the navigation symbols from the bottom of all slides uncomment this line
}

\usepackage{graphicx} % Allows including images
\usepackage{booktabs} % Allows the use of \toprule, \midrule and \bottomrule in tables

% New commands-environments
\newcommand{\eng}[1]{\selectlanguage{english}#1\selectlanguage{greek}}
\newcommand{\gre}[1]{\selectlanguage{greek}#1\selectlanguage{english}}

%----------------------------------------------------------------------------------------
%	TITLE PAGE
%----------------------------------------------------------------------------------------

\title[Καταγραφή και Δυναμική Ανάλυση της Κίνησης του Ανθρώπου]{Καταγραφή και Δυναμική Ανάλυση της Κίνησης του Ανθρώπου} 

%\author{\eng{Stanev Dimitar}} % Your name
\institute[ΗΜ\&ΤΥ]{
{\large\textsc{Πανεπιστήμιο Πατρών \\
Τμήμα Ηλεκτρολόγων Μηχανικών και Τεχνολογίας Υπολογιστών}}\\[1cm]
{\large\eng{Stanev Dimitar}}\\
\medskip
\textit{\eng{jimstanev@gmail.com}} % Your email address
}
\date{\today} % Date, can be changed to a custom date

\begin{document}

\begin{frame}
\titlepage % Print the title page as the first slide
\end{frame}

\begin{frame}
\frametitle{Ατζέντα}
\tableofcontents
\end{frame}

%----------------------------------------------------------------------------------------
%	PRESENTATION SLIDES
%----------------------------------------------------------------------------------------


\section{Εισαγωγή}
\subsection{Σκοπός}
\begin{frame}
\frametitle{Σκοπός}

\end{frame}

\subsection{Παραδείγματα}
\begin{frame}
\frametitle{Ενδεικτικές Εφαρμογές}

\end{frame}

\section{Καταγραφή της Κίνησης}
\subsection{Η συσκευή}
\begin{frame}
\frametitle{Η συσκευή \eng{Kinect}}

\end{frame}

\subsection{Αλγόριθμος ανίχνευσης σκελετού}
\begin{frame}
\frametitle{Αλγόριθμος ανίχνευσης σκελετού}

\end{frame}

\subsection{Αντιμετώπιση θορύβου στις μετρήσεις}
\begin{frame}
\frametitle{Αντιμετώπιση θορύβου στις μετρήσεις}

\end{frame}

\section{Υλικά και Μέθοδοι}
\subsection{Περιγραφή της διαδικασίας}
\begin{frame}
\frametitle{Περιγραφή της διαδικασίας}

\end{frame}

\subsection{Περιγραφή του μοντέλου}
\begin{frame}
\frametitle{Περιγραφή του μοντέλου}

\end{frame}

\subsection{Αντίστροφη κινηματική}
\begin{frame}
\frametitle{Αντίστροφη κινηματική}

\end{frame}

\subsubsection{Τοποθέτηση ενδείξεων}
\begin{frame}
\frametitle{Τοποθέτηση ενδείξεων}

\end{frame}

\subsubsection{Κανονικοποίησης του μοντέλου}
\begin{frame}
\frametitle{Κανονικοποίησης του μοντέλου}

\end{frame}

\subsection{Αντίστροφη δυναμική}
\begin{frame}
\frametitle{Αντίστροφη δυναμική}

\end{frame}

\subsection{Υπολογισμός μυϊκών διεγέρσεων}
\begin{frame}
\frametitle{Υπολογισμός μυϊκών διεγέρσεων}

\end{frame}

\subsection{Ορθή δυναμική}
\begin{frame}
\frametitle{Ορθή δυναμική}

\end{frame}

\section{Αποτελέσματα}
\begin{frame}
\frametitle{Αποτελέσματα}

\end{frame}

\end{document} 