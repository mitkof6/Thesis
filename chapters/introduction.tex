%%%%%%%%%%%%%%%%%%%%%%%%%%%%%%%%%%%%%%%%%%%%%%%%%%%%%%%%%%%%%%%%%%%%%%%%%%%%%%%%
\chapter{Εισαγωγή}

Πολλοί παράγοντες συμβάλουν στην συντεταγμένη κίνηση του σώματος και η γοητεία των επιστημόνων έχει οδηγήσει σε αμέτρητα πειράματα και μελέτες ώστε να μπορούν να την εξηγήσουν. Ως αποτέλεσμα υπάρχει πληθώρα υλικό και αποτελέσματα από τις μελέτες των χαρακτηριστικών των μυών, την γεωμετρική τους συσχέτιση με τα οστά και το αποτέλεσμα της κίνησης των αρθρώσεων. Κατά καιρούς έχουν γίνει κλινικές μελέτες σε ασθένειες όπως είναι η εγκεφαλική παράλυση, το εγκεφαλικό επεισόδιο, η οστεοαρθρίτιδα, η Νόσο του Πάρκινσον, ώστε να μελετηθούν οι νευρικές διεγέρσεις που οδηγούν την κίνηση του σώματος τόσο πριν την θεραπεία αλλά και μετά, με σκοπό να εξαχθούν συμπεράσματα για την αντιμετώπιση τους. Δυστυχώς η σύνθεση των δεδομένων από τις κλινικές μελέτες για την κατανόηση της δυσλειτουργίας που οφείλεται στην ασθένεια και την δημιουργία μια επιστημονικής βάσης για την αντιμετώπιση της ανώμαλης κίνησης παραμένει μια σημαντική πρόκληση.

Η χρήση πειραμάτων για την κατανόηση της δυναμικής της κίνησης έχει κάποια μειονεκτήματα. Για παράδειγμα η εκτίμηση των δυνάμεων που παράγονται από τους μύες είναι ακατόρθωτο να μετρηθούν πειραματικά. Επίσης υπάρχει δυσκολία κατανόησης των φαινομένων δράσης-αντίδρασης σε τόσο πολύπλοκα συστήματα μόνο από πειράματα. Ο προσδιορισμός της συνεισφοράς κάθε μυ στην κίνηση δεν είναι προφανές γιατί πολλές φορές η δράση του δεν συνεπάγεται μόνο την επιτάχυνση της άρθρωσης στην οποία δρα \cite{zajac-gordon89}, αλλά και σε άλλες αρθρώσεις.

Απαραίτητη προϋπόθεση για την μελέτη πολύπλοκων διατάξεων είναι η ανάπτυξη θεωρητικών μοντέλων, που μπορούν να προσεγγίσουν τις ανθρώπινες δραστηριότητες και σε συνδυασμό με τις πειραματικές ενδείξεις για να γίνει συστηματική μελέτη. Το σύστημα θα πρέπει να είναι σε θέση να φανερώσει τις εξαρτήσεις μεταξύ του νευρικού, του μυϊκού, του σκελετικού συστήματος και της κίνησης του σώματος. Είναι φανερό ότι τα αποτελέσματα αυτού του είδους αναλύσεων παρέχουν μεγαλύτερη πληροφορία και δίνουν την δυνατότητα στους ερευνητές να εξάγουν κατάλληλη θεραπεία για την ασθένεια.

Τα τελευταία χρόνια και ιδιαίτερα με την ανάπτυξη των υπολογιστών ήμαστε σε θέση να εκτελέσουμε πολύπλοκες προσομοιώσεις σε μικρότερο χρονικό διάστημα (της τάξεως μερικών ωρών) που πριν μια δεκαετία θα χρειαζόταν και μέρες. Ο παράγοντας που βελτιώσε την δραματική μείωση του χρόνου δεν οφείλεται μόνο στην ανάπτυξη των υπολογιστών, αλλά και στην εφεύρεση νέων μεθόδων. Η δυναμική προσομοίωση, η ανάπτυξη μυοσκελετικών και νευρομυοσκελετικών μοντέλων είναι σε θέση να δώσουν λύση σε πολλά προβλήματα που απασχολούν την ιατρική και είναι ευρέως αποδεχτά και έχουν μελετηθεί σε βάθος \cite{thelen-chumanov06, piazza06, pandy01, zajac02}.

Παρόλα αυτά υπάρχουν πολλά προβλήματα και φαινόμενα που δεν έχουν μοντελοποιηθεί ώστε να δώσουν πρόγνωση. Τα μοντέλα που έχουν δημιουργηθεί δεν είναι τόσο ακριβής και υπάρχει περιθώριο βελτιώσεων. Επίσης κάποιες αναλύσεις εκτελούνται ακόμα με απαγορευτικούς χρόνους, κάνοντας τα ακατάλληλα πολλές φορές. Μεγάλη πρόοδος έχει γίνει στην ανάπτυξη αξιόπιστων μοντέλων για τον μυ, αλλά η μεγάλη διαφοροποίηση που υπάρχει στο ανθρώπινο σώμα καθιστά δύσκολη την χρήση ενός γενικού μοντέλου. Από την άλλη υπάρχει μεγάλο χάσμα στην διασύνδεση του νευρικού συστήματος με το κινητήριο σύστημα και τα μοντέλα που υπάρχουν είναι απλοποιημένα και δεν αναπαριστούν πλήρες τις λειτουργίες τους. Συμπερασματικά, πρέπει να γίνει πολύ δουλεία τόσο από την μεριά των επιστημόνων που μελετούν την ανθρώπινη φυσιολογία, αλλά και των μηχανικών που μοντελοποιούν τα συστήματα προσομοιώσεων.

Είναι αναγκαία η δημιουργία ενιαίων εργαλείων και μοντέλων που θα χρησιμοποιούνται από την επιστημονική κοινότητα στις μελέτες τους. Υπάρχουν πολλά εργαστήρια που έχουν αναπτύξει ενδιαφέρουσες τεχνικές, ωστόσο τα αποτελέσματα τους δεν είναι διαθέσιμα για να χρησιμοποιηθούν από τρίτους. Υπάρχουν αρκετά εμπορικά εργαλεία (\eng{Anybody, Adams, Visuals 3D}) που είναι κλειστού κώδικα και δεν δίνουν την δυνατότητα επέκτασης και ευελιξίας. Η ανάγκη αυτή έχει οδηγήσει στην ανάπτυξη του ανοιχτού συστήματος \eng{OpenSim} που υποστηρίζεται από μια μεγάλη κοινότητα με πολλά μοντέλα και αναλυτικές μεθόδους, έχοντας βοηθήσει σημαντικά τους ερευνητές τα τελευταία χρόνια. Δίνοντας την δυνατότητα στους επιστήμονες να μοιράζονται κοινές μεθόδους ανάλυσης, μοντέλα και αποτελέσματα, βοηθάει στην βελτίωση, στην αναπαραγωγή και στην επιβεβαίωση των αποτελεσμάτων που έχουν εξαχθεί από τρίτους.

%%%%%%%%%%%%%%%%%%%%%%%%%%%%%%%%%%%%%%%%%%%%%%%%%%%%%%%%%%%%%%%%%%%%%%%%%%%%%%%%
\section{Εφαρμογές}

Ως κλασικό παράδειγμα θα αποδείξουμε ότι τα μυοσκελετικά μοντέλα μπορούν να χρησιμοποιηθούν για να εκτιμηθεί η πορεία θεραπείας ορθοπεδικών εγχειρήσεων. Η βάδιση με λυγισμένα τα γόνατα (\eng{crouch gait}) είναι μια από τις πιο κοινές ανωμαλίες σε άτομα με εγκεφαλική παράλυση. Χαρακτηριστικό αυτής της βάδισης είναι η ανικανότητα να παραχθεί αρκετή δύναμη από τους μύες στο γόνατο ώστε να σηκωθεί η λεκάνη (μια κίνηση που απαιτεί πολύ δύναμη και κόπωση από τους μύες).  Μια αιτία είναι το μικρό μήκος του ημιτενοντώδη μυ και μερικές φορές η θεραπεία είναι η επιμήκυνση του. Ωστόσο, μπορούν να υπάρχουν και άλλες αιτίες της υπερβολικής κάμψης του γόνατος (π.χ. αδύναμοι καμπτήρες της ποδοκνημικής άρθρωσης), και η επιμήκυνση του ημιτενοντώδη μπορεί να θέσει σε κίνδυνο την αντοχή των μυών στον αστράγαλο \cite{arnolda06}.

\begin{figure}[H]
    \centering
    \includegraphics[width=0.8\textwidth, keepaspectratio]{fig/crouch-gait.png}
    \caption{Πρόβλημα περπατήματος με λυγισμένα γόνατα \cite{arnolda06}}
    \label{fig:crouch-gait}
\end{figure}

Για αυτό το λόγο μπορούν να γίνουν καταγραφές της βάδισης του ασθενή σε ειδικά εργαστήρια και σε συνδυασμό με κατάλληλο μοντέλο των κάτω άκρων, ώστε να μελετηθεί για τον συγκεκριμένο ασθενή αν πρέπει να γίνει η επιμήκυνση του ημιτενοντώδη. Στα μοντέλα είναι δυνατή η παραμετροποίηση των μυών για το συγκεκριμένο ασθενή ώστε να μελετηθεί με ορθή δυναμική το αποτέλεσμα της εγχείρισης εικονικά. Τέτοιου είδους αναλύσεις θα ήταν ένα χρήσιμο εργαλείο για τους ιατρούς ώστε να τους δώσουν μια εποπτική κατάσταση του ασθενή, να βελτιώσουν την επιτυχία της εγχείρισης αλλά και την δημιουργία μιας θεραπευτικής αγωγής.

Ως δεύτερο παράδειγμα \cite{fregly07} είναι η μελέτη τρόπου βαδίσματος ώστε να μειωθεί η καταπόνηση του γονάτου. Σε αυτή την μελέτη οι ασθενείς με προβλήματα οστεοαρθρίτιδα καταπονούν το γόνατο κατά την βάδιση, οπότε οι συγγραφείς προτείνουν μια πολύ απλή παραλλαγή της βάδισης ώστε να μειωθεί η δύναμη που ασκείται στο γόνατο, στρέφοντας ελαφρά το πόδι προστάξω. Κατά το πείραμα οι ασθενείς καταγράφονται από συστήματα παρακολούθησης της κίνησης και επίσης καταγράφεται η αντίδραση εδάφους. Εφαρμόζεται αντίστροφη κινηματική, αντίστροφη δυναμική και σε πραγματικό χρόνο υπάρχει μέτρηση της δύναμης που ασκείται στο γόνατο, αλλά και του προσανατολισμού του ποδιού. Στα πλαίσια του πειράματος έχει αναπτυχθεί μια συσκευή που σηματοδοτεί τον ασθενή όταν δεν ακολουθεί τους κανόνες της βάδισης για την μείωση της καταπόντισης. Ως αποτέλεσμα μετά από τέσσερα σεμινάρια οι ασθενείς είχαν συνηθίσει στο νέο τρόπο βάδισης ώστε να μην καταπονούν το γόνατο.

\begin{figure}[H]
    \centering
    \includegraphics[width=0.8\textwidth, keepaspectratio]{fig/knee-load.png}
    \caption{Αποτελέσματα της ροπής της προσαγωγή στο γόνατο \cite{fregly07}}
    \label{fig:knee-load}
\end{figure}

Σαν τρίτο παράδειγμα, θα ήταν ενδιαφέρον να μπορούσαμε να μελετήσουμε την συμπεριφορά των φαρμάκων στις ασθένειες κατά την δραστηριότητα των ανθρώπων. Ως βασικό προαπαιτούμενο απαιτείται η κατανόηση και η μοντελοποίηση της δράσης του φαρμάκου ώστε να μπορεί να προσομοιωθεί. Επίσης απαιτείται η μοντελοποίηση της ασθένειας και το πως αυτή συνδέεται με τις δραστηριότητες του ανθρώπου. Έχουν γίνει μελέτες μοντελοποίησης φαρμάκων όπως είναι η ντοπαμίνη για την θεραπεία της Νόσος του Πάρκινσον \cite{haeri05}. Ωστόσο, υπάρχουν δυσκολίες όπως είναι η μοντελοποίηση του βασικά γάγγλια (\eng{basal ganglia}). Παρόλο αυτά μπορούν να εξαχθούν ενδιαφέροντα αποτελέσματα και να εξηγηθούν πράγματα που ίσως δώσουν λύση στην εύρεση μεθόδων θεραπείας δύσκολων ασθενειών.

Συμπερασματικά, οι προσομοιώσεις μπορούν να βοηθήσουν την επιστήμη της ιατρικής και όχι μόνο στο να μελετά την συμπεριφορά μιας ασθένειας, αλλά και να εξάγει συμπεράσματα θεραπείας και στρατηγικές αγωγής. Οι μέθοδοι προσομοιώσεων χρησιμοποιούνται σε άλλες βιομηχανίες, όπως είναι η βιομηχανία των αυτοκινήτων για την σχεδίαση κινητήρων και έχει μειώσει δραματικά το κόστος κατασκευής τους. Ως εκ τούτου θα πρέπει να χρησιμοποιούνται στην ιατρική, στην παραγωγή νέων φαρμάκων, την δημιουργία νέων ορθοπεδικών μηχανημάτων όπως είναι οι υποστηρικτές σκελετικού συστήματος \cite{stopforth12} και σε πολλούς άλλους κλάδους.

%%%%%%%%%%%%%%%%%%%%%%%%%%%%%%%%%%%%%%%%%%%%%%%%%%%%%%%%%%%%%%%%%%%%%%%%%%%%%%%%
\section{Σχετική Βιβλιογραφία}

Η εκτίμηση των μυϊκών δυνάμεων κατά την διεξαγωγή μιας κίνησης είναι αδύνατων να καταγραφούν εύκολα πειραματικά σε κλινικές μελέτες. Η μεγάλη ανακάλυψη τα τελευταία χρόνια είναι η δυνατότητα εκτίμησης της συνεισφοράς κάθε μυ σε συγκεκριμένη κίνηση απευθείας από την καταγεγραμμένη κίνηση, που είναι και ο κλασικός τρόπος διεξαγωγής αποτελεσμάτων \cite{hamner10, mclean03}. Η αντίστροφη δυναμική έχει γίνει ρουτίνα σε κλινικές αναλύσεις της βάδισης, υπολογίζοντας τις ροπές που ασκούνται στις αρθρώσεις κατά την δοσμένη κίνηση, ωστόσο απαιτεί γνώση των εξωτερικές δυνάμεων που ασκούνται στο σύστημα. Η μέθοδος της στατικής βελτιστοποίησης (\eng{static optimization}) σε συνδυασμό με τα αποτελέσματα της αντίστροφης δυναμικής είναι σε θέση να εκτιμήσει της συνεισφορά κάθε μυ \cite{heintz06, erdemir07} και χρησιμοποιείται εδώ και δεκαετίες.

Η δυσκολία απόκτησης πειραματικών δεδομένων αλλά και το σφάλμα που εισάγει η αντίστροφη δυναμική οδήγησε την ερευνητική κοινότητα σε εναλλακτικές μεθόδους με χρήση της ορθής δυναμικής \cite{buchanan04}. Παρόλο που η αντίστροφη δυναμική απαιτεί ολοκληρώσεις και γενικά είναι υπολογιστικά πιο ακριβή, έχουν αναπτυχθεί ενδιαφέρουσες μεθόδους ανάλυσης. Δεδομένα που μπορούν να χρησιμοποιηθούν και να βελτιώσουν το αποτέλεσμα της ορθής δυναμικής μπορούν να είναι οι καταγεγραμμένες διεγέρσεις μυών μέσω \eng{EMG}, μυικές δυνάμεις οι οποίες μπορούν να μετρηθούν με διάφορα όργανα και οι ροπές στις αρθρώσεις. Σε περίπτωση που δεν γνωρίζουμε τις εξωτερικές δυνάμεις που δρουν στην διάταξη, μπορούν να εισαχθούν πολύ εύκολα στις εξισώσεις ως περιορισμοί και να εκτιμηθούν κατά την διάρκεια των προσημειώσεων \cite{hamner10, seitha11}. Πολλές φορές δεν είναι γνωστές οι είσοδοι του συστήματος, οπότε σε τέτοιες περιπτώσεις διεγείρεται το σύστημα ελαφρός, εκτελείται η ορθή δυναμική ώστε να εκτιμηθεί το αποτέλεσμα και στην συνέχει γίνεται βελτιστοποίηση των εισόδων μέχρις ότου παραχθεί το επιθυμητό αποτέλεσμα \cite{pandy01}. Οι μέθοδοι αυτοί βασίζονται στην θεωρία του βέλτιστου έλεγχου και η επιλογή κατάλληλων κριτηρίων βελτιστοποίησης εξειδικεύεται ανάλογα με την εφαρμογή. Μια άλλη κατηγορία μεθόδων είναι τα παρακολούθησης (\eng{forward dynamics assisted data tracking}). Αρχικά τροφοδοτείται κατάλληλα το σύστημα της ορθής δυναμικής ώστε να εκτιμηθεί το αποτέλεσμα και σε συνδυασμό με ένα κλειστό βρόγχο γίνεται προσπάθεια να ελεγχθεί (μειωθεί) το σφάλμα μεταξύ της καταγεγραμμένης κίνησης και της εκτιμώμενης. Μια από τις καλύτερες μέθοδος που ανήκει σε αυτή την κατηγόρια είναι ο υπολογισμός μυϊκών διεγέρσεων (\eng{computed muscle control}) και προτάθηκε από \cite{thelen06}.

Απαραίτητη προϋπόθεση ώστε να γίνει η ανάλυση και η εκτίμηση των μυϊκών δυνάμεων είναι η ύπαρξη μυοσκελετικών μοντέλων που συσχετίζουν την σκελετική δομή με την παραγόμενη δύναμη από τους μύες. Απαραίτητο στοιχείο αυτής της δομής είναι η σωστή μοντελοποίηση της δυναμικής του μυ. Τα μοντέλα μυών που χρησιμοποιούνται συνήθως στις προσομοιώσεις είναι τύπου \eng{Hill} και προτάθηκαν αρχικά από \cite{zajac89}, στην συνέχεια με μικρές τροποποίησες από \cite{thelen03} και το η πιο πρόσφατη εργασία \cite{millard13}. Η τελευταία επιβεβαίωσε πειραματικά τα αποτελέσματα των μοντέλων και πρότεινε εναλλακτικό μοντέλο το οποίο βελτιώνει κατά πολύ τον χρόνο εκτίμησης των δυνάμεων και συνεπώς τον χρόνο που απαιτεί η όλη διαδικασία, χωρίς να χαθεί μειωθεί η ακρίβεια των υπολογισμών. Από την άλλη είναι αναγκαία η διασύνδεση των μυϊκών δυνάμεων με τις δυναμικές εξισώσεις κίνησης και για αυτό απαιτείται η εισαγωγή της μυϊκή ροπή αδράνειας (\eng{muscle moment arm}) \cite{delp95}, η οποία συνδέει την ροπή της άρθρωσης με τους μύες που δρουν σε αυτήν. Τέλος, αφού υπάρχουν τα μαθηματικά εργαλεία το επόμενο βήμα είναι η γεωμετρική τοποθέτηση των μυών πάνω στο σκελετικό σύστημα με βάση την φυσιολογία του ανθρώπου.

Η καταγραφή της κίνησης είναι ένα δύσκολο πρόβλημα και υπάρχουν πληθώρες λύσεις ανάλογα με την εφαρμογή. Ενδιαφέρον παρουσιάζει η εκτίμηση της θέσης των αρθρώσεων στο τρισδιάστατο χώρο, γιατί η παρακολούθηση της κίνησης περιγράφεται αποτελεσματικά από την παρακολούθηση των επιμέρους αρθρώσεων \cite{poppe07}. Οι αλγόριθμοι εκτίμησης της θέσης των αρθρώσεων συνήθως χρησιμοποιούν χάρτες βάθους που παρέχονται από συσκευές στερεοσκοπικής όρασης. Ένα από τους καλύτερους αλγορίθμους εκτίμησης της θέσης των αρθρώσεων \cite{shotton11}, όσον αφορά την ταχύτητα παροχής αποτελεσμάτων (εφαρμογή πραγματικού χρόνου) και μορφολογική διαφοροποίηση ανθρώπινων χαρακτηριστικών είναι υλοποιημένος εσωτερικά στην συσκευή \eng{Kinect}. Το \eng{Kinect} είναι μια φθηνή συσκευή στερεοσκοπικής όρασης που είναι σε θέση να καταγράψει τις ανθρώπινες δραστηριότητες και όχι μόνο.

Η επιστημονική κοινότητα έχει αναπτύξει ενδιαφέρουσες μεθόδους ανάλυση της δυναμικής συμπεριφοράς του ανθρώπου, ωστόσο πολλές φορές τα αποτελέσματα των ερευνών δεν είναι διαθέσιμα σε τρίτους. Επίσης υπάρχουν πολλές εμπορικές εφαρμογές που σου παρέχουν την τεχνογνωσία, αλλά ωστόσο είναι ακριβά και δεν παρέχουν την δυνατότητα προσαρμογή στις απαιτήσεις του χρήστη. Για το λόγο αυτό αναπτύχθηκε τα τελευταία χρόνια μια βιβλιοθήκη-εφαρμογή για την διεξαγωγή των αναλύσεων, συγκεντρώνοντας τις πιο πρόσφατες ανακαλύψεις μεθόδων \cite{delp07}.

%%%%%%%%%%%%%%%%%%%%%%%%%%%%%%%%%%%%%%%%%%%%%%%%%%%%%%%%%%%%%%%%%%%%%%%%%%%%%%%%
\section{Συνεισφορά}

Η συνεισφορά της παρούσας διπλωματικής εργασίας είναι αρχικά, η διεξαγωγή της καταγραφής της ανθρώπινης κίνησης με σχετικά φθηνό και αποτελεσματικό τρόπο με χρήση της συσκευής \eng{Kinect}. Λόγο του θορύβου της καταγεγραμμένης κίνησης μελετούνται μέθοδοι ώστε να βελτιωθεί το αποτέλεσμα και να διωχθεί ο θόρυβος, βήμα απαραίτητο για την διεξαγωγή σωστών αποτελεσμάτων στα μετέπειτα στάδια. Στη συνέχει εστιαζόμαστε στην μοντελοποίηση των κάτω άκρων του ανθρώπινου σώματος με χρήση μυοσκελετικών μοντέλων ώστε να ήμαστε σε θέσει να απαντήσουμε στα ερωτήματα που αφορούν την ανθρώπινη βάδιση. Το μοντέλο μυών που χρησιμοποιήθηκε είναι τύπου \eng{Hill} και βασίζεται στην πρόσφατη εργασία \cite{millard13}. Το μοντέλο των κάτω άκρων αποτελείται από 86 μύες. Λόγο αδυναμίας καταγραφής των εξωτερικών δυνάμεων (δυνάμεις αντίδρασης εδάφους) κατά την βάδισης εξετάζεται η εισαγωγή περιορισμών ή δυνάμεις επαφής και μελετάται η ορθότητα τους με βάση πειραματικών δεδομένων από τρίτους.

Στο κομμάτι της ανάλυσης αρχικά αφού έχει καταγραφεί η κίνηση και έχει γίνει η κατάλληλη αντιστοίχηση μεταξύ των αρθρώσεων της κίνησης και του μοντέλου. Έπειτα κανονικοποιείται του γενικού μοντέλου ώστε να ταιριάζει στο πειραματικό δείγμα, μειώνοντας κατά πολύ το σφάλμα της αντίστροφης κινηματικής. Στην συνέχει διεξάγεται αντίστροφη κινηματική ώστε να προσδιοριστούν οι γενικευμένες συντεταγμένες του μοντέλου για την δοσμένη κίνηση. Αρχικά εξετάζεται η συνέπεια της διαδικασία της αντίστροφης δυναμικής ως μέθοδο υπολογισμού των γενικευμένων ροπών ή δυνάμεων στις αρθρώσεις. Στην συνέχει προσδιορίζονται οι μυϊκές διεγέρσεις των μυών με κριτήριο την ελάχιστη ενέργεια για την διεξαγωγή της δοσμένης κίνησης. Οι υπολογισμένες διεγέρσεις τροφοδοτούνται στο σύστημα ανοιχτού βρόγχου και μελετάται η ευστάθεια του συστήματος με χρήση ορθής δυναμικής.

Η διάρθρωση της διπλωματικής αρχικά ξεκινά με την περιγραφή των χαρακτηριστικών του αισθητήρα, πώς μπορούμε να έχουμε πρόσβαση στα δεδομένα που μας ενδιαφέρουν, τα διάφορα εργαλεία που είναι διαθέσιμα, αλλά και τρόποι μείωσης του θορύβου των μετρήσεων. Στο τρίτο κεφάλαιο γίνεται μια εισαγωγή στις βασικές έννοιες της ρομποτικής, δίνοντας μαθηματικούς ορισμούς που είναι απαραίτητοι για την κατασκευή της διάταξης που θα μελετήσουμε. Στο τέταρτο κεφάλαιο αναφερόμαστε στην μοντελοποίηση των μυών, εστιάζοντας στα μοντέλα \eng{Hill-Type} και εξηγούνται τα στάδια από την στιγμή της παραγωγής νευρικών διεγέρσεων έως την δημιουργία της κίνησης. Στο πέμπτο κεφάλαιο μιλάμε για την ροή της ανάλυσης που ακολουθήθηκε και προτείνουμε μια στρατηγική ώστε να επιτευχθούν ορθά αποτελέσματα. Τέλος, στο έκτο κεφάλαιο παρουσιάζονται τα αποτελέσματα των πειραμάτων.

