%%%%%%%%%%%%%%%%%%%%%%%%%%%%%%%%%%%%%%%%%%%%%%%%%%%%%%%%%%%%%%%%%%%%%%%%%%%%%%%%
\section*{Περίληψη}

Αντικείμενο της παρούσας διπλωματικής εργασίας είναι αρχικά η καταγραφή της ανθρώπινης κίνησης με κάποια συσκευή παρακολούθησης και κατόπιν η δημιουργία ενός αντιπροσωπευτικού μοντέλου, ώστε να μπορεί να μελετηθεί η δυναμική του συμπεριφορά. Ως συσκευή καταγραφής χρησιμοποιήθηκε ο αισθητήρας \eng{Kinect} της \eng{Microsoft}. Το μοντέλο που αναπτύχθηκε αφορά κυρίως τα κάτω άκρα του ανθρώπου και επιπλέον διαθέτει μυοσκελετική δομή με 86 μύες. Στα πλαίσια των αναλύσεων χρησιμοποιήθηκαν διάφορες τεχνικές για την εξαγωγή των αποτελεσμάτων, όπως είναι η αντίστροφη κινηματική, αντίστροφη δυναμική, υπολογισμός μυϊκών διεγέρσεων και ορθή δυναμική και προτείνουμε μια στρατηγική για την ανάλυση και την εξαγωγή αποτελεσμάτων.

Στο πρώτο κεφάλαιο γίνεται μια αναφορά στο πώς οι μεθοδολογίες που αναπτύχθηκαν στα πλαίσια της διπλωματικής εργασίας μπορούν να χρησιμοποιηθούν σε πραγματικές εφαρμογές, την σχετική βιβλιογραφία και την συνεισφορά της παρούσας εργασίας. Στο δεύτερο κεφάλαιο γίνεται μια ανασκόπηση των χαρακτηριστικών του αισθητήρα, της πρόσβασης στα δεδομένα που μας ενδιαφέρουν, των διαφόρων διαθέσιμων εργαλείων, αλλά και τρόπων μείωσης του θορύβου των μετρήσεων, πράγμα απαραίτητο για την εξαγωγή έγκυρων αποτελεσμάτων στα μετέπειτα στάδια επεξεργασίας. Στο τρίτο κεφάλαιο γίνεται μια εισαγωγή στις βασικές έννοιες της ρομποτικής, δίνοντας μαθηματικούς ορισμούς που είναι απαραίτητοι για την κατανόηση και μοντελοποίηση της διάταξης που θα μελετηθεί. Στο τέταρτο κεφάλαιο γίνεται αναφορά στη μοντελοποίηση των μυών, εστιάζοντας στα μοντέλα τύπου \eng{Hill} και εξηγείται η διαδικασία από την στιγμή της παραγωγής νευρικών διεγέρσεων έως τη δημιουργία της κίνησης. Στο πέμπτο κεφάλαιο συνοψίζεται η ροή της ανάλυσης που ακολουθήθηκε και προτείνεται μια στρατηγική για την επίτευξη ορθών αποτελεσμάτων. Τέλος, στο έκτο κεφάλαιο παρουσιάζονται τα αποτελέσματα των πειραμάτων.

\vfill

\paragraph{\textbf{Λέξεις κλειδιά:}}Καταγραφή της κίνησης, ορθή δυναμική, αντίστροφη δυναμική, υπολογισμός μυϊκών διεγέρσεων, μυοσκελετικά μοντέλα

\thispagestyle{empty}
\clearpage\mbox{}
\thispagestyle{empty}
\clearpage

%%%%%%%%%%%%%%%%%%%%%%%%%%%%%%%%%%%%%%%%%%%%%%%%%%%%%%%%%%%%%%%%%%%%%%%%%%%%%%%%
\section*{\texorpdfstring{\eng{Abstract}}{}}

\en
The research developed in this thesis first deals with the problem of capturing the human body motion and then concentrates on the creation of musculoskeletal model to study its dynamical behavior. The Microsoft's Kinect sensor was used to capture the human motion. The model used for the simulations is the part of the human lower limb with 86 attached muscles. For the analysis phase we used some common methods such as inverse kinematics, inverse dynamics, computed muscle control and forward dynamics and we showed a general pipeline strategy for generating correct results.

The first chapter describes how the analysis methodology developed in this thesis can be used in real applications. The second chapter contains information about the motion capture sensor, its characteristics, how to access its data, the alternative tools to program it and a way to deal with noisy measurements. In the third chapter serves as an introduction to some robotics concepts, which are necessary for the development of the model. The fourth chapter provides the Hill-Type muscle model and explains how a muscle can generate force from neural excitation input. The fifth chapter presents the methods used to analyse the captured data. Finally, our experimental evaluation are shown.
\gr

\vfill

\paragraph{\textbf{\eng{Keywords:}}}\eng{Motion capture, forward dynamics, inverse dynamics, computed muscle control, musculoskeletal models}

\thispagestyle{empty}
\clearpage\mbox{}
\thispagestyle{empty}
