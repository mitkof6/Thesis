%%%%%%%%%%%%%%%%%%%%%%%%%%%%%%%%%%%%%%%%%%%%%%%%%%%%%%%%%%%%%%%%%%%%%%%%%%%%%%%%
\section*{Περίληψη}

Αντικείμενο της παρούσας διπλωματικής εργασίας είναι αρχικά η καταγραφή της κίνησης του ανθρώπου και κατόπιν η δημιουργία ενός αντιπροσωπευτικού μοντέλου, ώστε να μπορεί να μελετηθεί η δυναμική του συμπεριφορά. Ως συσκευή καταγραφής χρησιμοποιήθηκε ο αισθητήρας \eng{Kinect} της \eng{Microsoft}. Το μοντέλο που αναπτύχθηκε αφορά κυρίως τα κάτω άκρα του ανθρώπου και επιπλέον διαθέτει μυοσκελετική δομή με 86 μύες. Στα πλαίσια των αναλύσεων χρησιμοποιήθηκαν διάφορες τεχνικές για την εξαγωγή των αποτελεσμάτων, όπως είναι η αντίστροφη κινηματική, αντίστροφη δυναμική, υπολογισμός μυϊκών διεγέρσεων και ορθή δυναμική και προτείνουμε μια στρατηγική για την ανάλυση και την εξαγωγή αποτελεσμάτων.

Στο πρώτο κεφάλαιο γίνεται μια αναφορά στο πώς οι μεθοδολογίες που αναπτύχθηκαν στα πλαίσια της διπλωματικής εργασίας μπορούν να χρησιμοποιηθούν σε πραγματικές εφαρμογές. Στο δεύτερο κεφάλαιο γίνεται μια ανασκόπηση των χαρακτηριστικών του αισθητήρα, πώς μπορούμε να έχουμε πρόσβαση στα δεδομένα που μας ενδιαφέρουν, τα διάφορα εργαλεία που είναι διαθέσιμα, αλλά και τρόποι μείωσης του θορύβου των μετρήσεων. Στο τρίτο κεφάλαιο γίνεται μια εισαγωγή στις βασικές έννοιες της ρομποτικής, δίνοντας μαθηματικούς ορισμούς που είναι απαραίτητοι για την κατασκευή της διάταξης που θα μελετήσουμε. Στο τέταρτο κεφάλαιο αναφερόμαστε στην μοντελοποίηση των μυών, εστιάζοντας στα μοντέλα \eng{Hill-Type} και εξηγούνται τα στάδια από την στιγμή της παραγωγής νευρικών διεγέρσεων έως την δημιουργία της κίνησης. Στο πέμπτο κεφάλαιο μιλάμε για την ροή της ανάλυσης που ακολουθήθηκε και προτείνουμε μια στρατηγική ώστε να επιτευχθούν ορθά αποτελέσματα. Τέλος, στο έκτο κεφάλαιο παρουσιάζονται τα αποτελέσματα των πειραμάτων.

\vfill

\paragraph{\textbf{Λέξεις κλειδιά:}}Καταγραφή κίνησης, ορθή δυναμική, αντίστροφη δυναμική, υπολογισμός μυϊκών διεγέρσεων, μυοσκελετικά μοντέλα

\thispagestyle{empty}
\clearpage\mbox{}
\thispagestyle{empty}
\clearpage

%%%%%%%%%%%%%%%%%%%%%%%%%%%%%%%%%%%%%%%%%%%%%%%%%%%%%%%%%%%%%%%%%%%%%%%%%%%%%%%%
\section*{\texorpdfstring{\eng{Abstract}}{}}

\en
The research developed in this thesis first deals with the problem of capturing the human body motion and then concentrates on the creation of musculoskeletal model to study its dynamics behavior. The Microsoft's Kinect sensor was used to capture the human motion. The model used for the simulations is the part of the human lower limb with 86 attached muscles. For the analysis phase we used some common methods such as inverse kinematics, inverse dynamics, computed muscle control and forward dynamics and we showed a general pipeline strategy for generating correct results.

The first chapter describes the usage of the methodology analysis developed in this thesis and its applications in real life problems. The second chapter discuss about the motion capture sensor, its characteristics, how to access its data, the alternative tools to programm it and a way to deal with noisy measurements. In the third chapter an introduction to some robotics concepts is made, which is necessary for the development of the model. The forth chapter provides the Hill-Type muscle model and explains how a muscle can generate force from neural excitation input. The fifth chapter states the methods used to analyse the captured data. Finally, we present our experimental evaluation.
\gr

\vfill

\paragraph{\textbf{\eng{Keywords:}}}\eng{Motion capture, forward dynamics, inverse dynamics, computed muscle control, musculoskeletal models}

\thispagestyle{empty}
\clearpage\mbox{}
\thispagestyle{empty} 